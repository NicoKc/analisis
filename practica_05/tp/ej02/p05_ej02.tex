\documentclass[../practica_05.tex]{subfiles}

\begin{document}

    \begin{enumerate}
        \item 
            \begin{itemize}
                \item $f(x) = ln(x+1)^2$
                \item orden 3
                \item $x_0 = 0$
            \end{itemize}

            $ $

            \begin{itemize}
                \item $f(x) = \ln(x+1)^2 \Rightarrow f(0) = 0$
                \item $f^\prime(x) = 2\ln(x+1)\frac{1}{x+1} \Rightarrow f^\prime(0) = 0$
                \item $f^{\prime\prime}(x) = \frac{2}{(x+1)^2} - \frac{2\ln(x+1)}{(x+1)^2} \Rightarrow f^{\prime\prime}(0) = 2$
                \item $f^{\prime\prime\prime}(x) = \frac{4\ln(x+1)-6}{(x+1)^3} \Rightarrow f^{\prime\prime\prime}(0) = -6$
            \end{itemize}

            $\Rightarrow P_3(x) = 2\frac{x^2}{2!} - 6 \frac{x^3}{3!} = $
            $ x^2 - x^3 $

        \item 
            \begin{itemize}
                \item $g(x) = e^{x+2}$
                \item orden 3
                \item $x_0 = 0$
            \end{itemize}

            \begin{itemize}
                \item $g(x) = e^{x+2} \Rightarrow g(0) = e^2$
                \item $g^\prime(x) = e^{x+2} \Rightarrow g^\prime(0) = e^2$
                \item $g^{\prime\prime}(x) = e^{x+2} \Rightarrow g^{\prime\prime}(0) = e^2$
                \item $g^{\prime\prime\prime}(x) = e^{x+2} \Rightarrow g^{\prime\prime\prime}(0) = e^2$
            \end{itemize}

            $\Rightarrow P_3(x) = e^2 (1 + x + \frac{x^2}{2} + \frac{x^3}{6})$

        \item

            \begin{itemize}
                \item $p(x) = x^4 - 5x^3 + 5x^2 + x + 2$
                \item potencias de $x-2$
            \end{itemize}

            \begin{itemize}
                \item $p(x) = x^4 - 5x^3 + 5x^2 + x + 2$
                \item orden 4
                \item $x_0 = 2$
            \end{itemize}

            \begin{itemize}
                \item $p(x) = x^4 - 5x^3 + 5x^2 + x + 2 \Rightarrow p(2) = 16 - 5\cdot 8 + 5\cdot 4 + 2 + 2 = 0$
                \item $p^\prime(x) = 4x^3 - 15x^2 + 10x + 1 \Rightarrow p^\prime(2) = -7 $
                \item $p^{\prime\prime}(x) = 12x^2 - 30x + 10 \Rightarrow p^{\prime\prime}(2) = -2$
                \item $p^{\prime\prime\prime}(x) = 24x - 30 \Rightarrow p^{\prime\prime\prime}(2) = 18$
                \item $p^{\prime\prime\prime\prime}(x) = 24 \Rightarrow p^{\prime\prime\prime}(2) = 24$
            \end{itemize}

            $\Rightarrow P_4(x) = -7(x-2) - 2\frac{(x-2)^2}{2!} + 18\frac{(x-2)^3}{3!} + 24\frac{(x-2)^4}{4!} = $
            $ -7(x-2) - (x-2)^2 + 3(x-2)^3 + (x-2)^4 $

        \item

            \begin{itemize}
                \item $g(x) = \sqrt[]{x}$
                \item potencias de $x-1$
                \item orden 3
                \item $x_0 = 1$
            \end{itemize}

            \begin{itemize}
                \item $g(x) = \sqrt{x}                                          \Rightarrow g                     (1) = 1 $
                \item $g^\prime(x) = \frac{1}{2\sqrt{x}}                        \Rightarrow g^\prime              (1) = \frac{1}{2} $
                \item $g^{\prime\prime}(x) = \frac{-1}{4x^{\frac{3}{2}}}        \Rightarrow g^{\prime\prime}      (1) = -\frac{1}{4}$
                \item $g^{\prime\prime\prime}(x) = \frac{3}{8x^{\frac{5}{2}}}   \Rightarrow g^{\prime\prime\prime}(1) = \frac{3}{8}$
            \end{itemize}

            $\Rightarrow P_4(x) = 1 + \frac{1}{2}(x-1) - \frac{1}{4}\frac{(x-1)^2}{2!} + \frac{3}{8}\frac{(x-1)^3}{3!} = $
            $1 + \frac{(x-1)}{2} - \frac{(x-1)^2}{8} + \frac{(x-1)^3}{16} = $

    \end{enumerate}

\end{document}

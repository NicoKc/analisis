\documentclass[../practica_09.tex]{subfiles}

\begin{document}

    \begin{enumerate}
        \item $\int\int_D x^2ydA, D:$ mitad superior del disco con centro en el origen y radio 5
        
            \begin{itemize}
                \item $x = r\cos(\theta)$
                \item $y = r\sin(\theta)$
                \item $0\leq r \leq 5$
                \item $0\leq \theta \leq \pi$
            \end{itemize} $\Rightarrow$

            $T:\mathbb{R}^2 \to \mathbb{R}^2$

            $T(r,\theta) = (r\cos(\theta),r\sin(\theta)) $

            $\int\int_D f(x,y) dA(x,y) = \int \int_D f(T(r,\theta)) \abs*{JT(u,v)} dA(u,v)$

            \begin{itemize}
                \item $f(T(r,\theta)) = r^2\cos^2(\theta)r\sin(\theta) = $
                    $r^3\cos^2(\theta)\sin(\theta)$
            \end{itemize}

            $JT(u,v) = \begin{vmatrix}
                \cos(\theta)    -r\sin(\theta)
                \sin(\theta)    r\cos(\theta)
            \end{vmatrix} = r\cos^2(\theta) + r\sin^2(\theta) = r$

            $\int_0^{\pi} (\int_0^5 r^3\cos^2(\theta)\sin(\theta) dr) d\theta = $

            \begin{itemize}
                \item $\int_0^5 r^3\cos^2(\theta)\sin(\theta) dr = $
                
                    $\left. \frac{r^4\cos^2(\theta)\sin(\theta)}{4} \right |_0^5 = $

                    $ \frac{625\cos^2(\theta)\sin(\theta)}{4} $

                \item $\frac{625}{4} \int_0^{\pi} \cos^2(\theta)\sin(\theta) = $
                
                    $\frac{625}{4} \left. -\frac{1}{3} \cos^3(\theta) \right |_0^{\pi} = $

                    $ \frac{625}{4} \cdot \frac{2}{3}  $

            \end{itemize}

        \item $\int\int_D (2x-y) dA, D:$ region del primer cuadrante encerrada por la circunferencia $x^2 + y^2 = 4$, $x = 0$ e $y=x$
        
            $D = \{(r,\theta) \in \mathbb{R}^2: 0\leq r \leq 2 \wedge \frac{\pi}{4} \leq \frac{\pi}{2}\}$  

            $T(r,\theta) = (r\cos(\theta),r\sin(\theta))$

            $ \int_{\frac{\pi}{4}}^{\frac{\pi}{2}} ( \int_0^2 f(T(r,\theta)) \cdot JT(r,\theta) dr ) d\theta = $

            $ \int_{\frac{\pi}{4}}^{\frac{\pi}{2}} ( \int_0^2 2r\cos(\theta) - r\sin(\theta) \cdot r dr ) d\theta $

            \begin{itemize}
                \item $\int_0^2 2r\cos(\theta) - r\sin(\theta) \cdot r dr = $
                
                    $\int_0^2 2r^2\cos(\theta) - r^2\sin(\theta) dr = $

                    $\int_0^2 r^2(2\cos(\theta) - \sin(\theta)) dr = $

                    $(2\cos(\theta) - \sin(\theta)) \int_0^2 r^2 dr = $

                    $(2\cos(\theta) - \sin(\theta)) (\left. \frac{r^3}{3}\right |_0^2 )= $

                    $(2\cos(\theta) - \sin(\theta)) \frac{8}{3} = $

                    $(2\cos(\theta) - \sin(\theta)) \frac{8}{3} = $

                \item $\frac{8}{3} \int_{\frac{\pi}{4}}^{\frac{\pi}{2}} 2\cos(\theta) - \sin(\theta) d\theta = $
                
                    $\frac{8}{3} \left. -2\sin(\theta) - \cos(\theta) \right |_{\frac{\pi}{4}}^{\frac{\pi}{2}} = $

                    $\frac{8}{3} \cdot 2\left(1-\frac{1}{\sqrt{2}}\right)-\frac{1}{\sqrt{2}} $

            \end{itemize}


        \item $\int\int_D \sin(x^2+y^2) dA, D:$ region del primer cuadrante encerrada por la circunferencia con centro en el origen y radios 1 y 3
        
            $\int_0^{\frac{\pi}{2}} (\int_1^3 \sin(r^2) r dr) d\theta = $

            \begin{itemize}
                \item $\int_1^3 \sin(r^2) r dr = $
                
                    $\left. - \frac{1}{2}\cos(r^2) \right |_1^3 = $

                    $ -\frac{1}{2}\cos(9) + \frac{1}{2}\cos(1) =  $

                    $ \frac{1}{2}(\cos(1)- \cos(9)) $

                \item $\frac{1}{2}(\cos(1)- \cos(9)) \int_0^{\frac{\pi}{2}} 1 d\theta = $

                    $ \frac{1}{2}(\cos(1)- \cos(9)) (\frac{pi}{2})$

            \end{itemize}
        
        \item $\int\int_D e^{-x^2-y^2} dA, D:$ region acotada por las semicircunferencias $x=\sqrt{4-y^2}$ y el eje $y$
        
            $\int_{\frac{3\pi}{2}}^{\frac{5\pi}{2}} (\int_0^2 e^{-r^2} r dr) d\theta = $

            \begin{itemize}
                \item $\int_0^2 e^{-r^2} r dr = $
                
                    $ \left. -e^{-r^2} \right |_0^2 = $

                    $ -e^{-4} - 1 $

                \item $\int_{\frac{3\pi}{2}}^{\frac{5\pi}{2}} -e^{-4} - 1 d\theta = $
                
                    $ (-e^{-4} - 1) \int_{\frac{3\pi}{2}}^{\frac{5\pi}{2}} 1 d\theta = $

                    $ (-e^{-4} - 1)(\frac{5\pi}{2}- \frac{3\pi}{2} ) = $

                    $(-e^{-4} - 1)\pi$

            \end{itemize}

    \end{enumerate}

\end{document}

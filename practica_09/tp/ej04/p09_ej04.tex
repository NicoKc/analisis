\documentclass[../practica_09.tex]{subfiles}

\begin{document}

    \begin{enumerate}
        \item $ \int\int\int_D 1 dV, D=\{0\leq r \leq 2 \wedge 0 \leq \theta \leq 2\pi \wedge 0 \leq z \leq r \} $
        
        \item Bajo el paraboloide $z=18-2x^2-2y^2$ y arriba del plano $xy$

            $ 0 = 18 - (2(x^2 + y^2)) \equiv $

            $ 2(x^2+y^2) = 18 \equiv $

            $ x^2 + y^2 = 9 $ Circunferencia de radio 3 centrada en el origen

            Al sustituir por polares $D$ es de tipo 1

            En el plano $xy$ la sombra que se proyecta del parboloide es $ r\cdot \theta : (r,\theta) \in \mathbb{R}^3: 0 \leq r \leq 3 \wedge 0 \leq \theta \leq 2\pi \Rightarrow$

            \begin{itemize}
                \item $ 0 \leq r \leq 3 $
                \item $ 0 \leq \theta \leq 2\pi$
                \item $ 0 \leq z \leq 18 - 2(x^2+y^2) \stackrel{polares}{\equiv} 0 \leq z \leq 18 - 2r^2 $
            \end{itemize}

            $ \int_0^3 (\int_0^{2\pi} (\int_0^{18-2r^2} 1 dz) d\theta) dr $

            \begin{itemize}
                \item $ \int_0^{18-2r^2} r dz = $

                    $ \left. rz \right |_0^{18-2r^2} = $

                    $ r(18-2r^2) - 0 = $

                    $ 18r-2r^3 $

                \item $ \int_0^{2\pi} 18r-2r^3 d\theta = $

                    $\left. 18r\theta - 2r^3\theta \right |_0^{2\pi} = $

                    $ 18r\theta -  $

                    $ 18r\pi-4r^3\pi $

                \item $ \int_0^3 18r\pi-4r^3\pi dr = $

                    $ \left. 9r^2\pi - r^4\pi \right |_0^3 = $

                    $ 81\pi - 81\pi = $


            \end{itemize}

            $\int_0^3 (\int_0^{2\pi} 18r - 2r^3 d\theta) dr$

            \begin{itemize}
                \item $ \int_0^{2\pi} 18r - 2r^3 d\theta = $

                    $ 2( \int_0^{2\pi} 9r - r^3 d\theta )= $

                    $ 2(\left. 9r\theta - \theta r^3 \right |_0^{2\pi}) = $

                    $ 2 ( 18r\pi - 2\pi r^3) = $

                    $ 4 (9r\pi - \pi r^3) = $

                \item $4 ( \int_0^3 9r\pi - \pi r^3 dr) = $

                    $ 4 (\left. \frac{9\pi}{2}r^2 - \frac{\pi}{4}r^4 \right |_0^3) = $

                    $ 4 ( \frac{81\pi}{2} - \frac{81\pi}{4}) = $

                    $ 2\cdot 81\pi - 81\pi = 81\pi $

            \end{itemize}

        \item Encerrado por el hiperboloide $-x^2-y^2+z^2 = 1$ y el plano $z = 2$

            \begin{itemize}
                \item $ z^2 - 1 = x^2 + y^2 \stackrel{z=2}{\equiv} 3 = x^2 + y^2 $ Circunferencia de radio $\sqrt{3}$ centrado en $(0,0)$
                \item $ z = 2 $
            \end{itemize}

            $\int_0^{\sqrt{3}}(\int_0^{2\pi} 1 d\theta)dr$

            \begin{itemize}
                \item $\int_0^{2\pi} r d\theta = $

                    $ \left. r\theta \right |_0^{2\pi} = $

                    $ 2r\pi $

                \item $\int_0^{\sqrt{3}} 2r\pi dr =$
                
                    $\left. r^2\pi \right |_0^{\sqrt{3}} =$

                    $ 3\pi $

            \end{itemize}

        \item 1er octante = $ \int_2^4 (\int_0^{\frac{\pi}{2}} (\int_r^{\sqrt{16-r^2}} r dz ) d\theta) dr $

            \begin{itemize}
                \item $ \int_r^{\sqrt{16-r^2}} r dz = $
                
                    $\left. rz \right |_r^{\sqrt{16-r^2}} = $

                    $ r\sqrt{16-r^2} - r^2 $

                \item $ \int_0^{\frac{\pi}{2}} r\sqrt{16-r^2} - r^2 d\theta = $
                
                    $\left. r\sqrt{16-r^2}\theta - r^2\theta \right |_0^{\frac{\pi}{2}} = $

                    $\frac{r\sqrt{16-r^2}\pi}{2} - \frac{r^2\pi}{2}$

                \item $\int_2^4 \frac{r\sqrt{16-r^2}\pi}{2} - \frac{r^2\pi}{2} dr = $
                
                    \begin{itemize}
                        \item $\int \frac{r\sqrt{16-r^2}\pi}{2} dr = \frac{\pi}{2} \int r(16-r^2)^{\frac{1}{2}} dr = $
                        
                            $ \frac{(16-r^2)^{\frac{3}{2}}}{3} + C$

                            $ (\frac{(16-r^2)^{\frac{3}{2}}}{3})_r = \frac{1}{3} \cdot \frac{3(16-r^2)^{\frac{1}{2}}2r}{2}$

                        \item $\int \frac{r^2\pi}{2} dr = \frac{r^3\pi}{6} + C$
                    \end{itemize}

                    $\left. \frac{\pi (16-r^2)^{\frac{3}{2}}}{6} - \frac{r^3\pi}{6} \right |_2^4 = $

                    $ \frac{\pi (16-4^2)^{\frac{3}{2}}}{6} - \frac{4^3\pi}{6} - (\frac{\pi (16-2^2)^{\frac{3}{2}}}{6} - \frac{2^3\pi}{6}) = $

                    $ - \frac{32\pi}{3} - \frac{12^{\frac{3}{2}}\pi}{6} + \frac{4\pi}{3} = \star \Rightarrow$

                    $ V = 8 \cdot \star $


            \end{itemize}

    \end{enumerate}

\end{document}

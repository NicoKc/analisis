\documentclass[../parcial.tex]{subfiles}

\begin{document}

    $f(x,y) = \left\{
        \begin{array}{ll}
            \frac{x^2y^2-\sin(x^4)}{x^2 + \frac{1}{3} y^2} + 2 \qquad si (x,y) \neq (0,0) \\
            a \qquad si (x,y) = (0,0) 
        \end{array}
    \right. \Rightarrow$

    $ h(x,y) = \frac{x^2y^2-\sin(x^4)}{x^2 + \frac{1}{3} y^2} $

    QvQ $ \lim_{(x,y) \to (0,0)} h(x,y) = 0 $

    $ \exists g(x,y) : \lim_{(x,y) \to (0,0)} g(x,y) = 0 \wedge 0 \leq \abs*{h(x,y)} \leq \abs*{g(x,y)}$

    $ \abs*{f(x,y)} = \abs*{\frac{x^2y^2-\sin(x^4)}{x^2 + \frac{1}{3} y^2}} = $

    $ \frac{\abs*{x^2y^2-\sin(x^4)}}{x^2 + \frac{1}{3} y^2} \leq $

    $ \frac{x^2y^2+\abs*{\sin(x^4)}}{x^2 + \frac{1}{3} y^2} \stackrel{\abs*{\sin(k)} \leq \abs*{k}}{\leq} $

    $ \frac{x^2y^2+x^4}{x^2 + \frac{1}{3} y^2} = $

    $ \frac{x^2}{x^2 + \frac{1}{3} y^2}y^2+x^4 \stackrel{\frac{x^2}{x^2 + \frac{1}{3} y^2} \leq 1}{\leq} $

    $ y^2+x^4 \stackrel{(x,y) \to 0 }{\to } 0$

    $ \Rightarrow \lim_{(x,y) \to (0,0)} h(x,y) = 0 $

    $ \Rightarrow f(x,y) $ es continua en todo $\real^2 \Leftrightarrow a = 2$ ya que es un conciente de polinomios y trigonometricas continuas donde el denominador se anula en el $(x,y) = (0,0)$

    $ $

    $ f(x,y) $ es diferencibale en el $(0,0) \Leftrightarrow$

    $ \exists L :\ lim_{(x,y) \to (0,0)} \frac{f(x,y) - f(0,0) - \grad f(0,0)\cdot(x,y) }{\norm*{(x,y)}} = L \wedge L = 0 $

    $ \grad f(x,y) = (f_x(x,y), f_y(x,y)) $

    \begin{itemize}
        \item $f_x(0,0)$
            $ f_x(0,0) = \lim_{h \to 0} \frac{f(h,0) - f(0,0)}{h} $
        
            $ f_x(0,0) = \lim_{h \to 0} \frac{\frac{-\sin(h^4)}{h^2} + 2 - 2}{h} = $
        
            $ f_x(0,0) = \lim_{h \to 0} \frac{-\sin(h^4)}{h^2}\cdot\frac{1}{h} = $
        
            $ f_x(0,0) = \lim_{h \to 0} \frac{-\sin(h^4)}{h^3} \stackrel{"\cdot \frac{h}{h}"}{=} $
        
            $ f_x(0,0) = \lim_{h \to 0} \frac{-h\sin(h^4)}{h^4} \stackrel{"\cdot \frac{h}{h}"}{=} $
        
            $ f_x(0,0) = \lim_{h \to 0} \frac{-h\sin(h^4)}{h^4} \stackrel{"lim_{k \to 0} \frac{\sin(k)}{k} = 1"}{=} $
        
            $ f_x(0,0) = \lim_{h \to 0} 1\cdot(-h) = 0 $

        \item $f_y(0,0)$
        
            $ f_y(0,0) = \lim_{h \to 0} \frac{f(0,h) - f(0,0)}{h} = $

            $ \lim_{h \to 0} \frac{\frac{0}{\frac{1}{3} h^2} + 2 - 2}{h} = $

            $ \lim_{h \to 0} \frac{0}{h} = 0 $

    \end{itemize}

    $\Rightarrow \grad f(0,0) = (0,0)$

    $ \Rightarrow  lim_{(x,y) \to (0,0)} \frac{f(x,y) - f(0,0) - \grad f(0,0)\cdot(x,y) }{\norm*{(x,y)}} = $

    $ \Rightarrow  lim_{(x,y) \to (0,0)} \frac{f(x,y) - f(0,0) - (0,0)\cdot(x,y) }{\norm*{(x,y)}} = $

    $ \Rightarrow  lim_{(x,y) \to (0,0)} \frac{f(x,y) - f(0,0)}{\norm*{(x,y)}} = $

    $ \Rightarrow  lim_{(x,y) \to (0,0)} \frac{\frac{x^2y^2-\sin(x^4)}{x^2 + \frac{1}{3} y^2} + 2 - 2}{\norm*{(x,y)}} = $

    $ \Rightarrow  lim_{(x,y) \to (0,0)} \frac{\frac{x^2y^2-\sin(x^4)}{x^2 + \frac{1}{3} y^2}}{\norm*{(x,y)}} = $

    $ \Rightarrow  lim_{(x,y) \to (0,0)} \frac{x^2y^2-\sin(x^4)}{x^2 + \frac{1}{3} y^2} \cdot\frac{1}{\norm*{(x,y)}} = $

    $ \Rightarrow  lim_{(x,y) \to (0,0)} \frac{x^2y^2-\sin(x^4)}{x^2 + \frac{1}{3} y^2} \cdot\frac{1}{\norm*{(x,y)}} = $

    $ \abs*{\frac{x^2y^2-\sin(x^4)}{x^2 + \frac{1}{3} y^2} \cdot\frac{1}{\norm*{(x,y)}}} = $

    $ \frac{\abs*{x^2y^2-\sin(x^4)}}{x^2 + \frac{1}{3} y^2} \cdot\frac{1}{\norm*{(x,y)}} \stackrel{"des.\ triang"}{\leq} $

    $ \frac{x^2y^2+\abs*{\sin(x^4)}}{x^2 + \frac{1}{3} y^2} \cdot\frac{1}{\norm*{(x,y)}} \stackrel{\abs*{\sin(k)} \leq \abs*{k}}{\leq} $

    $ \frac{x^2(y^2+x^2)}{x^2 + \frac{1}{3} y^2} \cdot\frac{1}{\norm*{(x,y)}} = $

    $ \frac{x^2}{x^2 + \frac{1}{3} y^2} \cdot (y^2+x^2) \cdot\frac{1}{\norm*{(x,y)}} \stackrel{\frac{x^2}{x^2 + \frac{1}{3} y^2} \leq 1}{\leq} $

    $ \frac{y^2+x^2}{\norm*{(x,y)}} \stackrel{\frac{x^2}{x^2 + \frac{1}{3} y^2} \leq 1}{\leq} $

    $ \frac{y^2+x^2}{\norm*{(x,y)}} = $

    $ \frac{\norm*{(x,y)}^{\cancel{2}}}{\cancel{\norm*{(x,y)}}} = $

    $ \norm*{(x,y)} \wedge \norm*{(x,y)} \stackrel{(x,y) \to (0,0)}{\to} 0 $

    $\Rightarrow $ entonces $f$ es diferencibale en (0,0)

\end{document} 

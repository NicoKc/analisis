\documentclass[../practica_03.tex]{subfiles}

\begin{document}

    \begin{enumerate}
        \item $f(x,y) = \frac{xy}{1+e^{x-y}}$

            Ptos criticos $ 1 + e^{x - y} = 0 \Leftrightarrow$

            $ e^{x - y} = -1\ \nexists\ (x,y) \in \real^2 \Rightarrow$

            \begin{itemize}
                \item $xy$ es un polinomio, por lo tanto es continua en $(x,y) \in \real^2$
                \item $1 + e^{x - y}$ es continua en todo $(x,y) \in \real^2$
                \item $\frac{xy}{1+e^{x-y}}$ es continua en todo $(x,y) \in \real^2$ ya que el denominador no se anula
            \end{itemize}

        \item $f(x,y) = \frac{1+x^2+y^2}{1-x^2-y^2}$

            $f$ es continua en todos los puntos donde el denominador no se anula

            $1-x^2-y^2 = 0 \equiv 1 - (x^2 + y^2) = 0 \Leftrightarrow x^2 + y^2 = 1 \Rightarrow$

            $f$ es continua en $\{(x,y) \in \real^2 : x^2+y^2 \neq 1 \}$

            $ $

        \item $f(x,y) = \left\{
            \begin{array}{ll}
                \frac{x^2y^3}{2x^2 +y^2} \qquad si\ (x,y)\neq(0,0)\\
                1 \qquad si\ (x,y) = (0,0)
            \end{array}
        \right.$

            $\frac{x^2y^3}{2x^2 +y^2}$ es continua en todos los puntos donde el denominador no se anula

            $ 2x^2 +y^2 = 0 \Leftrightarrow x = y = 0 $

            $f$ es continua en el (0,0) $\Leftrightarrow$

            \begin{enumerate}
                \item $(0,0) \in Dom(f)\ \checked$
                \item $\exists \lim_{(x,y)\to (0,0)} f(x,y)\ \checked$
                \item $\lim_{(x,y)\to (0,0)} f(x,y) = f(0,0)\ \crossproduct $
            \end{enumerate}

            Pruebo por curvas

            \begin{itemize}
                \item $x = 0$
                    $\lim_{y\to 0} \frac{0\cdot y^3}{y^2} = 0$
            \end{itemize}

            Por el iterado x=0, si el limite existe es 0 que es distinto de 1,
            Por lo tanto f es continua en todo su dominio menos el (0,0)

            $ $

        \item $f(x,y) = \left\{
            \begin{array}{ll}
                \frac{xy}{x^2 + xy +y^2} \qquad si\ (x,y)\neq(0,0)\\
                0 \qquad si\ (x,y) = (0,0)
            \end{array}
        \right.$

            $\frac{xy}{x^2 + xy +y^2}$ es continua en todos los puntos donde el denominador no se anula

            $ x^2 + xy +y^2 = 0 \Leftrightarrow x = y = 0 $

            $f$ es continua en el (0,0) $\Leftrightarrow$

            \begin{enumerate}
                \item $(0,0) \in Dom(f)\ \checked$
                \item $\exists \lim_{(x,y)\to (0,0)} f(x,y)\ \crossproduct$
                \item $\lim_{(x,y)\to (0,0)} f(x,y) = f(0,0) = 0\ \crossproduct $
            \end{enumerate}

            Pruebo por curvas

            \begin{itemize} %\frac{xy}{x^2 + xy +y^2}%
                \item $x = 0 \Rightarrow$
                    $\lim_{y\to 0} \frac{0\cdot y}{y^2} = 0$
                \item $y = 0 \Rightarrow$
                    $\lim_{x\to 0} \frac{x\cdot 0}{x^2} = 0$
                \item $y = x$

                    $\lim_{x\to 0} \frac{x^2}{3x^2} = \frac{1}{3}$
            \end{itemize}

            El limite no existe

        $ $

        \item $f(x,y) = \left\{
            \begin{array}{ll}
                \frac{xy^2-\sin(x^2y)}{\frac{1}{2}x^2 +y^2} \qquad si\ (x,y)\neq(0,0)\\
                0 \qquad si\ (x,y) = (0,0)
            \end{array}
        \right.$

        $\frac{xy^2-\sin(x^2y)}{\frac{1}{2}x^2 +y^2}$ es continua en todos los puntos donde el denominador no se anula

        $ \frac{1}{2}x^2 +y^2 = 0 \Leftrightarrow x = y = 0 $

        $f$ es continua en el (0,0) $\Leftrightarrow$

        \begin{enumerate}
            \item $(0,0) \in Dom(f)\ \checked$
            \item $\exists \lim_{(x,y)\to (0,0)} f(x,y)\ \checked$
            \item $\lim_{(x,y)\to (0,0)} f(x,y) = f(0,0) = 0\ \checked $
        \end{enumerate}

        Pruebo por curvas

        \begin{itemize} %\frac{xy^2-\sin(x^2y)}{\frac{1}{2}x^2 +y^2}%
            \item $x = 0 \Rightarrow$
                $\lim_{y\to 0} \frac{\sin(0)}{y^2} = 0$
            \item $y = 0 \Rightarrow$
                $\lim_{x\to 0} \frac{\sin(0)}{\frac{1}{2}x^2} = 0$
        \end{itemize}

        $L = 0$ Es candidato

        $ \exists \delta : \norm*{(x,y)} < \delta \Rightarrow \abs*{f(x,y)} < \epsilon $

        $ \abs*{\frac{xy^2-\sin(x^2y)}{\frac{1}{2}x^2 +y^2}} \leq $
        $ \frac{\abs*{x}y^2+\abs*{\sin(x^2y)}}{\frac{1}{2}x^2 +y^2} \leq$
        $ \frac{\abs*{x}y^2+x^2\abs*{y}}{\frac{1}{2}x^2 +y^2} \leq$
        $ \frac{\abs*{x}y^2+x^2\abs*{y}}{\frac{1}{2}(x^2 +y^2)} \leq $

        $ \frac{2\norm*{(x,y)}^{\cancel{3}}}{\frac{1}{2}\cancel{\norm*{(x,y)}^2}} \leq $
        $ 4\delta < \epsilon \Leftrightarrow delta < \frac{\epsilon}{4} $

    \end{enumerate}

\end{document} 

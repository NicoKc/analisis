\documentclass[../practica_04.tex]{subfiles}

\begin{document}

    $ f(x,y) = \left\{
    \begin{array}{ll}
        \frac{x^3y}{x^6+y^2} \qquad si (x,y) \neq (0,0)\\
        0 \qquad si (x,y) = (0,0)\\
    \end{array}
    \right.$

    QvQ $f$ tiene derivadas direccionales para todo $v \in \real^2$ en el origen pero no es continua

    $v = (v_1,v_2) \wedge \norm*{v} = 1 $

    Pruebo por definicion de derivadas direccionales

    $\lim_{h \to 0} \frac{f(hv_1,hv_2) - f(0,0)}{h} $

    $\lim_{h \to 0} \frac{\frac{(hv_1)^3(hv_2)}{(hv_1)^6+(hv_1)^2} - 0}{h} = $

    $\lim_{h \to 0} \frac{1}{h} \cdot \frac{(hv_1)^3(hv_2)}{(hv_1)^6+(hv_1)^2} = $

    $\lim_{h \to 0} \frac{1}{h} \cdot \frac{h^4v_1^3v_2}{h^6v_1^6+h^2v_1^2} = $

    $\lim_{h \to 0} \frac{1}{h} \cdot \frac{h^{2\cdot\cancel{2}}v_1^3v_2}{\cancel{h^2}(h^4v_1^6+v_1^2)} = $

    $\lim_{h \to 0} \cancel{\frac{1}{h}} \cdot \frac{h^{\cancel{2}}v_1^3v_2}{h^4v_1^6+v_1^2} = \frac{v_1^3\cdot v_2}{v_1^2} = v_1\cdot v_2  $

    Pruebo limites por la curva $ y = mx^3, m \in \real$

    $\lim_{x \to 0} f(x,mx^3) = $

    $\lim_{x \to 0} \frac{x^3mx^3}{x^6+m^2x^6} = $

    $\lim_{x \to 0} \frac{m\cancel{x^6}}{(m^2+1)\cancel{x^6}} = \frac{m}{m^2+1} $

    si $m \neq 0 \Rightarrow \lim_{x \to 0} f(x,mx^3) \neq 0 \Rightarrow \nexists lim_{(x,y)\to(0,0)} f(x,y)$ 

    Por lo tango $f$ no es continua en el origen

\end{document}
